\hypertarget{README_8md_source}{\section{R\+E\+A\+D\+M\+E.\+md}
}

\begin{DoxyCode}
00001 # The Mimetic Methods Toolkit (MTK)
00002 
00003 By: **Eduardo J. Sanchez, Ph.D. - esanchez at mail dot sdsu dot edu**
00004 
00005 ## 1. Description
00006 
00007 We define numerical methods that are based on discretizations preserving the
00008 properties of their continuum counterparts to be **mimetic**.
00009 
00010 The **Mimetic Methods Toolkit (MTK)** is a C++ library for mimetic numerical
00011 methods. It is arranged as a set of classes for **mimetic quadratures**,
00012 **mimetic interpolation**, and **mimetic finite differences** methods for the
00013 numerical solution of ordinary and partial differential equations.
00014 
00015 ## 2. Dependencies
00016 
00017 This README assumes all of these dependencies are installed in the following
00018 folder:
00019 
00020 ```
00021 $(HOME)/Libraries/
00022 ```
00023 
00024 In this version, the MTK optionally uses ATLAS-optimized BLAS and LAPACK
00025 routines for the internal computation on some of the layers. However, ATLAS
00026 requires both BLAS and LAPACK in order to create their optimized distributions.
00027 Therefore, the following dependencies tree arises:
00028 
00029 ### For Linux:
00030 
00031 1. LAPACK - Available from: http://www.netlib.org/lapack/
00032   1. BLAS - Available from: http://www.netlib.org/blas/
00033 
00034 2. GLPK - Available from: https://www.gnu.org/software/glpk/
00035 
00036 3. (Optional) ATLAS - Available from: http://math-atlas.sourceforge.net/
00037   1. LAPACK - Available from: http://www.netlib.org/lapack/
00038     1. BLAS - Available from: http://www.netlib.org/blas
00039 
00040 4. (Optional) Valgrind - Available from: http://valgrind.org/
00041 
00042 5. (Optional) Doxygen - Available from http://www.stack.nl/~dimitri/doxygen/
00043 
00044 ### For OS X:
00045 
00046 1. GLPK - Available from: https://www.gnu.org/software/glpk/
00047 
00048 ## 3. Installation
00049 
00050 ### PART 1. CONFIGURATION OF THE MAKEFILE.
00051 
00052 The following steps are required to build and test the MTK. Please use the
00053 accompanying `Makefile.inc` file, which should provide a solid template to
00054 start with. The following command provides help on the options for make:
00055 
00056 ```
00057 $ make help
00058 -----
00059 Makefile for the MTK.
00060 
00061 Options are:
00062 - all: builds the library, the tests, and examples.
00063 - mtklib: builds the library.
00064 - test: builds the test files.
00065 - example: builds the examples.
00066 
00067 - testall: runs all the tests.
00068 
00069 - gendoc: generates the documentation for the library.
00070 
00071 - clean: cleans all the generated files.
00072 - cleanlib: cleans the generated archive and object files.
00073 - cleantest: cleans the generated tests executables.
00074 - cleanexample: cleans the generated examples executables.
00075 -----
00076 ```
00077 
00078 ### PART 2. BUILD THE LIBRARY.
00079 
00080 ```
00081 $ make
00082 ```
00083 
00084 If successful you'll read (before building the tests and examples):
00085 
00086 ```
00087 ----- Library created! Check in /home/ejspeiro/Dropbox/MTK/lib
00088 ```
00089 
00090 ## 4. Contact, Support, and Credits
00091 
00092 The MTK is developed by researchers and adjuncts to the
00093 [Computational Science Research Center (CSRC)](http://www.csrc.sdsu.edu/)
00094 at [San Diego State University (SDSU)](http://www.sdsu.edu/).
00095 
00096 Developers are members of:
00097 
00098 1. Mimetic Numerical Methods Research and Development Group.
00099 2. Computational Geoscience Research and Development Group.
00100 3. Ocean Modeling Research and Development Group.
00101 
00102 Currently the developers are:
00103 
00104 - **Eduardo J. Sanchez, Ph.D. - esanchez at mail dot sdsu dot edu** - @ejspeiro
00105 - Jose E. Castillo, Ph.D. - jcastillo at mail dot sdsu dot edu
00106 - Guillermo F. Miranda, Ph.D. - unigrav at hotmail dot com
00107 - Christopher P. Paolini, Ph.D. - paolini at engineering dot sdsu dot edu
00108 - Angel Boada.
00109 - Johnny Corbino.
00110 - Raul Vargas-Navarro.
00111 
00112 Finally, please feel free to contact me with suggestions or corrections:
00113 
00114 **Eduardo J. Sanchez, Ph.D. - esanchez at mail dot sdsu dot edu** - @ejspeiro
00115 
00116 Thanks and happy coding!
\end{DoxyCode}
