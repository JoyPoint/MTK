\hypertarget{README_8md_source}{\section{R\+E\+A\+D\+M\+E.\+md}
}

\begin{DoxyCode}
00001 # The Mimetic Methods Toolkit (MTK)
00002 
00003 By: **Eduardo J. Sanchez, PhD - esanchez at mail dot sdsu dot edu**
00004 
00005 ## 1. Description
00006 
00007 We define numerical methods that are based on discretizations preserving the
00008 properties of their continuous counterparts to be **mimetic**.
00009 
00010 The **Mimetic Methods Toolkit (MTK)** is a C++11 library for mimetic numerical
00011 methods. It is a set of classes for **mimetic interpolation**, **mimetic
00012 quadratures**, and **mimetic finite difference** methods for the **numerical
00013 solution of ordinary and partial differential equations**.
00014 
00015 ## 2. Dependencies
00016 
00017 This README file assumes all of these dependencies are installed in the
00018 following folder:
00019 
00020 ```
00021 $(HOME)/Libraries/
00022 ```
00023 
00024 In this version, the MTK optionally uses ATLAS-optimized BLAS and LAPACK
00025 routines for the internal computation on some of the layers. However, ATLAS
00026 requires both BLAS and LAPACK in order to create their optimized distributions.
00027 Therefore, the following dependencies tree arises:
00028 
00029 ### For Linux:
00030 
00031 1. LAPACK - Available from: http://www.netlib.org/lapack/
00032   1. BLAS - Available from: http://www.netlib.org/blas/
00033 
00034 2. GLPK - Available from: https://www.gnu.org/software/glpk/
00035 
00036 3. (Optional) ATLAS - Available from: http://math-atlas.sourceforge.net/
00037   1. LAPACK - Available from: http://www.netlib.org/lapack/
00038     1. BLAS - Available from: http://www.netlib.org/blas
00039 
00040 4. (Optional) Valgrind - Available from: http://valgrind.org/
00041 
00042 5. (Optional) Doxygen - Available from http://www.stack.nl/~dimitri/doxygen/
00043 
00044 ### For OS X:
00045 
00046 1. GLPK - Available from: https://www.gnu.org/software/glpk/
00047 
00048 ## 3. Installation
00049 
00050 ### PART 1. CONFIGURATION OF THE MAKEFILE.
00051 
00052 The following steps are required to build and test the MTK. Please use the
00053 accompanying `Makefile.inc` file, which should provide a solid template to
00054 start with. The following command provides help on the options for make:
00055 
00056 ```
00057 $ make help
00058 -----
00059 Makefile for the MTK.
00060 
00061 Options are:
00062 - all: builds the library, the tests, and examples.
00063 - mtklib: builds the library.
00064 - test: builds the test files.
00065 - example: builds the examples.
00066 
00067 - testall: runs all the tests.
00068 
00069 - gendoc: generates the documentation for the library.
00070 
00071 - clean: cleans all the generated files.
00072 - cleanlib: cleans the generated archive and object files.
00073 - cleantest: cleans the generated tests executables.
00074 - cleanexample: cleans the generated examples executables.
00075 -----
00076 ```
00077 
00078 ### PART 2. BUILD THE LIBRARY.
00079 
00080 ```
00081 $ make
00082 ```
00083 
00084 If successful you'll read (before building the tests and examples):
00085 ```
00086 ----- Library created! Check in /home/ejspeiro/Dropbox/MTK/lib
00087 ```
00088 
00089 ## 4. Contact, Support, and Credits
00090 
00091 The GitHub repository is: https://github.com/ejspeiro/MTK
00092 
00093 The MTK is developed by researchers and adjuncts to the
00094 [Computational Science Research Center (CSRC)](http://www.csrc.sdsu.edu/)
00095 at [San Diego State University (SDSU)](http://www.sdsu.edu/).
00096 
00097 Currently the developers are:
00098 
00099 - **Eduardo J. Sanchez, PhD - esanchez at mail dot sdsu dot edu** - @ejspeiro
00100 - Jose E. Castillo, PhD - jcastillo at mail dot sdsu dot edu
00101 - Guillermo F. Miranda, PhD - unigrav at hotmail dot com
00102 - Christopher P. Paolini, PhD - paolini at engineering dot sdsu dot edu
00103 - Angel Boada.
00104 - Johnny Corbino.
00105 - Raul Vargas-Navarro.
00106 
00107 ### 4.1. Acknowledgements and Contributions
00108 
00109 The authors would like to acknowledge valuable advising, feedback,
00110 and actual contributions from research personnel at the Computational Science
00111 Research Center (CSRC) at San Diego State University (SDSU). Their input was
00112 important to the fruition of this work. Specifically, our thanks go to
00113 (alphabetical order):
00114 
00115 -# Mohammad Abouali, PhD
00116 -# Dany De Cecchis, PhD
00117 -# Otilio Rojas, PhD
00118 -# Julia Rossi.
00119 
00120 ## 5. Referencing This Work
00121 
00122 Please reference this work as follows:
00123 
00124 Please reference this work as follows:
00125 ```
00126 @article\{Sanchez2014308,
00127   title = "The Mimetic Methods Toolkit: An object-oriented \(\backslash\)\{API\(\backslash\)\} for Mimetic
00128 Finite Differences ",
00129   journal = "Journal of Computational and Applied Mathematics ",
00130   volume = "270",
00131   number = "",
00132   pages = "308 - 322",
00133   year = "2014",
00134   note = "Fourth International Conference on Finite Element Methods in
00135 Engineering and Sciences (FEMTEC 2013) ",
00136   issn = "0377-0427",
00137   doi = "http://dx.doi.org/10.1016/j.cam.2013.12.046",
00138   url = "http://www.sciencedirect.com/science/article/pii/S037704271300719X",
00139   author = "Eduardo J. Sanchez and Christopher P. Paolini and Jose E. Castillo",
00140   keywords = "Object-oriented development",
00141   keywords = "Partial differential equations",
00142   keywords = "Application programming interfaces",
00143   keywords = "Mimetic Finite Differences "
00144 \}
00145 
00146 @Inbook\{Sanchez2015,
00147   author="Sanchez, Eduardo and Paolini, Christopher and Blomgren, Peter
00148 and Castillo, Jose",
00149   editor="Kirby, M. Robert and Berzins, Martin and Hesthaven, S. Jan",
00150   chapter="Algorithms for Higher-Order Mimetic Operators",
00151   title="Spectral and High Order Methods for Partial Differential Equations
00152 ICOSAHOM 2014: Selected papers from the ICOSAHOM conference, June 23-27, 2014,
00153 Salt Lake City, Utah, USA",
00154   year="2015",
00155   publisher="Springer International Publishing",
00156   address="Cham",
00157   pages="425--434",
00158   isbn="978-3-319-19800-2",
00159   doi="10.1007/978-3-319-19800-2\_39",
00160   url="http://dx.doi.org/10.1007/978-3-319-19800-2\_39"
00161 \}
00162 ```
00163 
00164 Finally, please feel free to contact me with suggestions or corrections:
00165 
00166 **Eduardo J. Sanchez, PhD - esanchez at mail dot sdsu dot edu** - @ejspeiro
00167 
00168 Thanks and happy coding!
\end{DoxyCode}
