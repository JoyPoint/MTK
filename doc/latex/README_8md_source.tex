\hypertarget{README_8md_source}{\section{R\+E\+A\+D\+M\+E.\+md}
}

\begin{DoxyCode}
00001 # The Mimetic Methods Toolkit (MTK)
00002 
00003 By: **Eduardo J. Sanchez, Ph.D. - esanchez at mail dot sdsu dot edu**
00004     \_\_\_\_\_\_\_\_\_\_\_\_\_\_\_\_\_\_\_\_\_\_\_\_\_\_\_\_\_\_\_\_\_\_\_\_\_\_\_\_\_\_\_\_\_\_\_\_\_\_\_\_\_\_\_\_\_\_\_\_\_\_\_\_\_\_
00005 
00006 ## 1. Description
00007 
00008 We define numerical methods that are based on discretizations preserving the
00009 properties of their continuum counterparts to be **mimetic**.
00010 
00011 The **Mimetic Methods Toolkit (MTK)** is a C++ library for mimetic numerical
00012 methods. It is arranged as a set of classes for **mimetic quadratures**,
00013 **mimetic interpolation**, and **mimetic finite differences** methods for the
00014 numerical solution of ordinary and partial differential equations.
00015 
00016 An older version of this library is available outside of GitHub... just email me
00017 about it, and you can have it... it is ugly, yet it is functional and more
00018 complete.
00019     \_\_\_\_\_\_\_\_\_\_\_\_\_\_\_\_\_\_\_\_\_\_\_\_\_\_\_\_\_\_\_\_\_\_\_\_\_\_\_\_\_\_\_\_\_\_\_\_\_\_\_\_\_\_\_\_\_\_\_\_\_\_\_\_\_\_
00020 
00021 ## 2. Dependencies
00022 
00023 This README assumes all of these dependencies are installed in the following
00024 folder:
00025 
00026 ```
00027 $(HOME)/Libraries/
00028 ```
00029 
00030 In this version, the MTK optionally uses ATLAS-optimized BLAS and LAPACK
00031 routines for the internal computation on some of the layers. However, ATLAS
00032 requires both BLAS and LAPACK in order to create their optimized distributions.
00033 Therefore, the following dependencies tree arises:
00034 
00035 ### For Linux:
00036 
00037 1. LAPACK - Available from: http://www.netlib.org/lapack/
00038   1. BLAS - Available from: http://www.netlib.org/blas/
00039 
00040 2. GLPK - Available from: https://www.gnu.org/software/glpk/
00041 
00042 3. (Optional) ATLAS - Available from: http://math-atlas.sourceforge.net/
00043   1. LAPACK - Available from: http://www.netlib.org/lapack/
00044     1. BLAS - Available from: http://www.netlib.org/blas
00045 
00046 4. (Optional) Valgrind - Available from: http://valgrind.org/
00047 
00048 5. (Optional) Doxygen - Available from http://www.stack.nl/~dimitri/doxygen/
00049 
00050 ### For OS X:
00051 
00052 1. GLPK - Available from: https://www.gnu.org/software/glpk/
00053     \_\_\_\_\_\_\_\_\_\_\_\_\_\_\_\_\_\_\_\_\_\_\_\_\_\_\_\_\_\_\_\_\_\_\_\_\_\_\_\_\_\_\_\_\_\_\_\_\_\_\_\_\_\_\_\_\_\_\_\_\_\_\_\_\_\_
00054 
00055 ## 3. Installation
00056 
00057 ### PART 1. CONFIGURATION OF THE MAKEFILE.
00058 
00059 The following steps are required to build and test the MTK. Please use the
00060 accompanying `Makefile.inc` file, which should provide a solid template to
00061 start with. The following command provides help on the options for make:
00062 
00063 ```
00064 $ make help
00065 -----
00066 Makefile for the MTK.
00067 
00068 Options are:
00069 - all: builds the library, the tests, and examples.
00070 - mtklib: builds the library.
00071 - test: builds the test files.
00072 - example: builds the examples.
00073 
00074 - testall: runs all the tests.
00075 
00076 - gendoc: generates the documentation for the library.
00077 
00078 - clean: cleans all the generated files.
00079 - cleanlib: cleans the generated archive and object files.
00080 - cleantest: cleans the generated tests executables.
00081 - cleanexample: cleans the generated examples executables.
00082 -----
00083 ```
00084 
00085 ### PART 2. BUILD THE LIBRARY.
00086 
00087 ```
00088 $ make
00089 ```
00090 
00091 If successful you'll read (before building the tests and examples):
00092 
00093 ```
00094 ----- Library created! Check in /home/ejspeiro/Dropbox/MTK/lib
00095 ```
00096 
00097 Examples and tests will also be built.
00098     \_\_\_\_\_\_\_\_\_\_\_\_\_\_\_\_\_\_\_\_\_\_\_\_\_\_\_\_\_\_\_\_\_\_\_\_\_\_\_\_\_\_\_\_\_\_\_\_\_\_\_\_\_\_\_\_\_\_\_\_\_\_\_\_\_\_
00099 
00100 ## 4. Frequently Asked Questions
00101 
00102 Q: Why haven't you guys implemented GBS to build the library?
00103 A: I'm on it as we speak! ;)
00104 
00105 Q: Is there any main reference when it comes to the theory on Mimetic Methods?
00106 A: Yes! Check: http://www.csrc.sdsu.edu/mimetic-book
00107 
00108 Q: Do I need to generate the documentation myself?
00109 A: You can if you want to... but if you DO NOT want to, just go to our website.
00110     \_\_\_\_\_\_\_\_\_\_\_\_\_\_\_\_\_\_\_\_\_\_\_\_\_\_\_\_\_\_\_\_\_\_\_\_\_\_\_\_\_\_\_\_\_\_\_\_\_\_\_\_\_\_\_\_\_\_\_\_\_\_\_\_\_\_
00111 
00112 ## 5. Contact, Support, and Credits
00113 
00114 The MTK is developed by researchers and adjuncts to the
00115 [Computational Science Research Center (CSRC)](http://www.csrc.sdsu.edu/)
00116 at [San Diego State University (SDSU)](http://www.sdsu.edu/).
00117 
00118 Developers are members of:
00119 
00120 1. Mimetic Numerical Methods Research and Development Group.
00121 2. Computational Geoscience Research and Development Group.
00122 3. Ocean Modeling Research and Development Group.
00123 
00124 Currently the developers are:
00125 
00126 - **Eduardo J. Sanchez, Ph.D. - esanchez at mail dot sdsu dot edu** - @ejspeiro
00127 - Jose E. Castillo, Ph.D. - jcastillo at mail dot sdsu dot edu
00128 - Guillermo F. Miranda, Ph.D. - unigrav at hotmail dot com
00129 - Christopher P. Paolini, Ph.D. - paolini at engineering dot sdsu dot edu
00130 - Angel Boada.
00131 - Johnny Corbino.
00132 - Raul Vargas-Navarro.
00133 
00134 Finally, please feel free to contact me with suggestions or corrections:
00135 
00136 **Eduardo J. Sanchez, Ph.D. - esanchez at mail dot sdsu dot edu** - @ejspeiro
00137 
00138 Thanks and happy coding!
\end{DoxyCode}
