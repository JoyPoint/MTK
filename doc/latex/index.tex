We define numerical methods that are based on discretizations preserving the properties of their continuous counterparts to be {\bfseries mimetic}.

The {\bfseries Mimetic Methods Toolkit (M\+T\+K)} is a C++11 library for mimetic numerical methods. It is a set of classes for {\bfseries mimetic interpolation}, {\bfseries mimetic quadratures}, and {\bfseries mimetic finite difference} methods for the {\bfseries numerical solution of ordinary and partial differential equations}.\hypertarget{index_section_mtk_concerns}{}\section{M\+T\+K Concerns}\label{index_section_mtk_concerns}
Since collaborative development efforts are definitely important in achieving the level of generality we intend the library to possess, we have divided the library's source code according to the designated purpose the classes possess within the library. These divisions (or {\bfseries concerns}) are grouped by layers, and are hierarchically related by the dependence they have among them.

One concern is said to depend on another one, if the classes the first concern includes, rely on the classes the second concern includes.

In order of dependence these are\+:
\begin{DoxyEnumerate}
\item Roots.
\item Enumerations.
\item Tools.
\item Data Structures.
\item Numerical Methods.
\item Grids.
\item Mimetic Operators.
\end{DoxyEnumerate}\hypertarget{index_section_wrappers}{}\section{M\+T\+K Wrappers}\label{index_section_wrappers}
The M\+T\+K collection of wrappers is\+:


\begin{DoxyEnumerate}
\item M\+M\+T\+K\+: M\+A\+T\+L\+A\+B wrappers collection for M\+T\+K; intended for sequential computations.
\end{DoxyEnumerate}

Others are being strongly considered.\hypertarget{index_section_authors}{}\section{Contact, Support and Credits}\label{index_section_authors}
The Git\+Hub repository is\+: \href{https://github.com/ejspeiro/MTK}{\tt https\+://github.\+com/ejspeiro/\+M\+T\+K}

The M\+T\+K is developed by researchers and adjuncts to the \href{http://www.csrc.sdsu.edu/}{\tt Computational Science Research Center (C\+S\+R\+C)} at \href{http://www.sdsu.edu/}{\tt San Diego State University (S\+D\+S\+U)}.

Currently the developers are\+:


\begin{DoxyItemize}
\item {\bfseries Eduardo J. Sanchez, Ph\+D -\/ esanchez at mail dot sdsu dot edu} -\/ ejspeiro
\item Jose E. Castillo, Ph\+D -\/ jcastillo at mail dot sdsu dot edu
\item Guillermo F. Miranda, Ph\+D -\/ unigrav at hotmail dot com
\item Christopher P. Paolini, Ph\+D -\/ paolini at engineering dot sdsu dot edu
\item Angel Boada.
\item Johnny Corbino.
\item Raul Vargas-\/\+Navarro.
\end{DoxyItemize}\hypertarget{index_subsection_acknowledgements}{}\subsection{Acknowledgements and Contributions}\label{index_subsection_acknowledgements}
The authors would like to acknowledge valuable advising, contributions and feedback, from research personnel at the Computational Science Research Center at San Diego State University, which were vital to the fruition of this work. Specifically, our thanks go to (alphabetical order)\+:


\begin{DoxyEnumerate}
\item Mohammad Abouali, Ph.\+D.
\item Dany De Cecchis, Ph.\+D.
\item Otilio Rojas, Ph.\+D.
\item Julia Rossi. 
\end{DoxyEnumerate}