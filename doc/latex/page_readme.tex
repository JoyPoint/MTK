
\begin{DoxyPre}
\section*{README File for the Mimetic Methods Toolkit (MTK)}\end{DoxyPre}



\begin{DoxyPre}\end{DoxyPre}



\begin{DoxyPre}By: {\bfseries Eduardo J. Sanchez, Ph.D. - esanchez at mail dot sdsu dot edu}


\end{DoxyPre}



\begin{DoxyPre}\subsection*{1. Description}\end{DoxyPre}



\begin{DoxyPre}\end{DoxyPre}



\begin{DoxyPre}We define numerical methods that are based on discretizations preserving the
properties of their continuum counterparts to be {\bfseries mimetic}.\end{DoxyPre}



\begin{DoxyPre}The {\bfseries Mimetic Methods Toolkit (MTK)} is a C++ library for mimetic numerical
methods. It is arranged as a set of classes for {\bfseries mimetic quadratures},
{\bfseries mimetic interpolation}, and {\bfseries mimetic finite differences} methods for the
numerical solution of ordinary and partial differential equations.\end{DoxyPre}



\begin{DoxyPre}An older version of this library is available outside of GitHub... just email me
about it, and you can have it... it is ugly, yet functional and more complete.


\end{DoxyPre}



\begin{DoxyPre}\subsection*{2. Dependencies}\end{DoxyPre}



\begin{DoxyPre}\end{DoxyPre}



\begin{DoxyPre}This README assumes all of these dependencies are installed in the following
folder:\end{DoxyPre}



\begin{DoxyPre}\begin{DoxyVerb}$(HOME)/Libraries/
\end{DoxyVerb}
\end{DoxyPre}



\begin{DoxyPre}In this version, the MTK optionally uses ATLAS-optimized BLAS and LAPACK
routines for the internal computation on some of the layers. However, ATLAS
requires both BLAS and LAPACK in order to create their optimized distributions.
Therefore, the following dependencies tree arises:\end{DoxyPre}



\begin{DoxyPre}\subsubsection*{For Linux:}\end{DoxyPre}



\begin{DoxyPre}\end{DoxyPre}



\begin{DoxyPre}1. LAPACK - Available from: \href{http://www.netlib.org/lapack/}{\tt http://www.netlib.org/lapack/}
  1. BLAS - Available from: \href{http://www.netlib.org/blas/}{\tt http://www.netlib.org/blas/}\end{DoxyPre}



\begin{DoxyPre}2. GLPK - Available from: \href{https://www.gnu.org/software/glpk/}{\tt https://www.gnu.org/software/glpk/}\end{DoxyPre}



\begin{DoxyPre}3. (Optional) ATLAS - Available from: \href{http://math-atlas.sourceforge.net/}{\tt http://math-atlas.sourceforge.net/}
  1. LAPACK - Available from: \href{http://www.netlib.org/lapack/}{\tt http://www.netlib.org/lapack/}
    1. BLAS - Available from: \href{http://www.netlib.org/blas}{\tt http://www.netlib.org/blas}\end{DoxyPre}



\begin{DoxyPre}4. (Optional) Valgrind - Available from: \href{http://valgrind.org/}{\tt http://valgrind.org/}\end{DoxyPre}



\begin{DoxyPre}5. (Optional) Doxygen - Available from \href{http://www.stack.nl/~dimitri/doxygen/}{\tt http://www.stack.nl/~dimitri/doxygen/}\end{DoxyPre}



\begin{DoxyPre}\subsubsection*{For OS X:}\end{DoxyPre}



\begin{DoxyPre}\end{DoxyPre}



\begin{DoxyPre}1. GLPK - Available from: \href{https://www.gnu.org/software/glpk/}{\tt https://www.gnu.org/software/glpk/}


\end{DoxyPre}



\begin{DoxyPre}\subsection*{3. Installation}\end{DoxyPre}



\begin{DoxyPre}\end{DoxyPre}



\begin{DoxyPre}\subsubsection*{PART 1. CONFIGURATION OF THE MAKEFILE.}\end{DoxyPre}



\begin{DoxyPre}\end{DoxyPre}



\begin{DoxyPre}The following steps are required the build and test the MTK. Please use the
accompanying \hyperlink{Makefile_8inc}{Makefile.inc} file, which should provide a solid template to start
with. The following command provides help on the options for make:\end{DoxyPre}



\begin{DoxyPre}\begin{DoxyVerb}$ make help
-----
Makefile for the MTK.

Options are:
- all: builds the library, the tests, and examples.
- mtklib: builds the library.
- test: builds the test files.
- example: builds the examples.

- testall: runs all the tests.

- gendoc: generates the documentation for the library.

- clean: cleans all the generated files.
- cleanlib: cleans the generated archive and object files.
- cleantest: cleans the generated tests executables.
- cleanexample: cleans the generated examples executables.
-----
\end{DoxyVerb}
\end{DoxyPre}



\begin{DoxyPre}\subsubsection*{PART 2. BUILD THE LIBRARY.}\end{DoxyPre}



\begin{DoxyPre}\end{DoxyPre}



\begin{DoxyPre}\begin{DoxyVerb}$ make
\end{DoxyVerb}
\end{DoxyPre}



\begin{DoxyPre}If successful you'll read (before building the examples):\end{DoxyPre}



\begin{DoxyPre}\begin{DoxyVerb}----- Library created! Check in /home/ejspeiro/Dropbox/MTK/lib
\end{DoxyVerb}
\end{DoxyPre}



\begin{DoxyPre}Examples and tests will also be built.


\end{DoxyPre}



\begin{DoxyPre}\subsection*{4. Frequently Asked Questions}\end{DoxyPre}



\begin{DoxyPre}\end{DoxyPre}



\begin{DoxyPre}Q: Why haven't you guys implemented GBS to build the library?
A: I'm on it as we speak! ;)\end{DoxyPre}



\begin{DoxyPre}Q: Is there any main reference when it comes to the theory on Mimetic Methods?
A: Yes! Check: \href{http://www.csrc.sdsu.edu/mimetic-book}{\tt http://www.csrc.sdsu.edu/mimetic-book}\end{DoxyPre}



\begin{DoxyPre}Q: Do I need to generate the documentation myself?
A: You can if you want to... but if you DO NOT want to, just go to our website.


\end{DoxyPre}



\begin{DoxyPre}\subsection*{5. Contact, Support, and Credits}\end{DoxyPre}



\begin{DoxyPre}\end{DoxyPre}



\begin{DoxyPre}The MTK is developed by researchers and adjuncts to the
\href{http://www.csrc.sdsu.edu/}{\tt Computational Science Research Center (CSRC)}
at \href{http://www.sdsu.edu/}{\tt San Diego State University (SDSU)}.\end{DoxyPre}



\begin{DoxyPre}Developers are members of:\end{DoxyPre}



\begin{DoxyPre}1. Mimetic Numerical Methods Research and Development Group.
2. Computational Geoscience Research and Development Group.
3. Ocean Modeling Research and Development Group.\end{DoxyPre}



\begin{DoxyPre}Currently the developers are:\end{DoxyPre}



\begin{DoxyPre}
\begin{DoxyEnumerate}
\item {\bfseries Eduardo J. Sanchez, Ph.D. - esanchez at mail dot sdsu dot edu} - ejspeiro
\item Jose E. Castillo, Ph.D. - jcastillo at mail dot sdsu dot edu
\item Guillermo F. Miranda, Ph.D. - unigrav at hotmail dot com
\item Christopher P. Paolini, Ph.D. - paolini at engineering dot sdsu dot edu
\item Angel Boada.
\item Johnny Corbino.
\item Raul Vargas-Navarro.
\end{DoxyEnumerate}\end{DoxyPre}



\begin{DoxyPre}Finally, please feel free to contact me with suggestions or corrections:\end{DoxyPre}



\begin{DoxyPre}{\bfseries Eduardo J. Sanchez, Ph.D. - esanchez at mail dot sdsu dot edu} - ejspeiro\end{DoxyPre}



\begin{DoxyPre}Thanks and happy coding!
\end{DoxyPre}
 